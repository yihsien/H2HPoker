\usepackage{amsmath}    % need for subequations
\usepackage{graphicx}   % need for figures
\usepackage{verbatim}   % useful for program listings
\usepackage{color}      % use if color is used in text
\usepackage{subfig}	% use for side-by-side figures
\usepackage{hyperref}   % use for hypertext links, including those to external
                        % documents and URLs
\usepackage{soul}       % strike latex
\usepackage{amsthm}
\usepackage{amssymb}
\usepackage{wasysym}
\usepackage{relsize}

\newcommand{\eg}{{\em e.g.}, }
\newcommand{\ie}{{\em i.e.}, }
\newcommand{\etal}{{\em et al.\ }}
\newcommand{\etc}{{\em etc.}}
\newcommand{\eps}{{\epsilon}}
\newcommand{\mc}[1]{\mathcal{#1}}
\newcommand{\mb}[1]{\mathbb{#1}}
\newcommand{\euler}{\mathrm{e}}
\newcommand{\mat}[1]{{\bf#1}}
\newcommand{\Prod}{\mathlarger{\prod}}
\newcommand{\Sum}{\mathlarger{\sum}}
\newcommand{\eto}[1]{e^{#1}}

\renewcommand{\comment}[1]{}

\newtheorem*{theorem}{Theorem}
\newtheorem*{proposition}{Proposition}
\newtheorem*{statement}{Statement}

\theoremstyle{definition}
\newtheorem{example}{Example}[section]
\newtheorem*{counterexample}{Counter Example}

\allowdisplaybreaks

%% Aligned Box in equations of the align mode
\usepackage{calc,amsmath}
\newlength\dlf
\newcommand\alignedbox[2]{
  % #1 = before alignment
  % #2 = after alignment
  &
  \begingroup
  \settowidth\dlf{$\displaystyle #1$}
  \addtolength\dlf{\fboxsep+\fboxrule}
  \hspace{-\dlf}
  \boxed{#1 #2}
  \endgroup
}

%% underbar
\usepackage{accents}
\newcommand{\ubar}[1]{\underaccent{\bar}{#1}}
